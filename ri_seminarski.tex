% !TEX encoding = UTF-8 Unicode
\documentclass[a4paper]{article}
\usepackage{geometry}
 \geometry{
 a4paper,
 total={150mm,240mm},
 left=30mm,
 top=30mm,
 }

\usepackage{color}
\usepackage{url}
\usepackage[T2A]{fontenc} % enable Cyrillic fonts
\usepackage[utf8]{inputenc} % make weird characters work
\usepackage{graphicx}

\usepackage[english,serbian]{babel}
%\usepackage[english,serbianc]{babel} %ukljuciti babel sa ovim opcijama, umesto gornjim, ukoliko se koristi cirilica

\usepackage[unicode]{hyperref}
\hypersetup{colorlinks,citecolor=green,filecolor=green,linkcolor=blue,urlcolor=blue}

\usepackage{listings}

%\newtheorem{primer}{Пример}[section] %ćirilični primer
\newtheorem{primer}{Primer}[section]

\definecolor{mygreen}{rgb}{0,0.6,0}
\definecolor{mygray}{rgb}{0.5,0.5,0.5}
\definecolor{mymauve}{rgb}{0.58,0,0.82}

\lstset{ 
  backgroundcolor=\color{white},   % choose the background color; you must add \usepackage{color} or \usepackage{xcolor}; should come as last argument
  basicstyle=\scriptsize\ttfamily,        % the size of the fonts that are used for the code
  breakatwhitespace=false,         % sets if automatic breaks should only happen at whitespace
  breaklines=true,                 % sets automatic line breaking
  captionpos=b,                    % sets the caption-position to bottom
  commentstyle=\color{mygreen},    % comment style
  deletekeywords={...},            % if you want to delete keywords from the given language
  escapeinside={\%*}{*)},          % if you want to add LaTeX within your code
  extendedchars=true,              % lets you use non-ASCII characters; for 8-bits encodings only, does not work with UTF-8
  firstnumber=1000,                % start line enumeration with line 1000
  frame=single,	                   % adds a frame around the code
  keepspaces=true,                 % keeps spaces in text, useful for keeping indentation of code (possibly needs columns=flexible)
  keywordstyle=\color{blue},       % keyword style
  language=Python,                 % the language of the code
  morekeywords={*,...},            % if you want to add more keywords to the set
  numbers=left,                    % where to put the line-numbers; possible values are (none, left, right)
  numbersep=5pt,                   % how far the line-numbers are from the code
  numberstyle=\tiny\color{mygray}, % the style that is used for the line-numbers
  rulecolor=\color{black},         % if not set, the frame-color may be changed on line-breaks within not-black text (e.g. comments (green here))
  showspaces=false,                % show spaces everywhere adding particular underscores; it overrides 'showstringspaces'
  showstringspaces=false,          % underline spaces within strings only
  showtabs=false,                  % show tabs within strings adding particular underscores
  stepnumber=2,                    % the step between two line-numbers. If it's 1, each line will be numbered
  stringstyle=\color{mymauve},     % string literal style
  tabsize=2,	                   % sets default tabsize to 2 spaces
  title=\lstname                   % show the filename of files included with \lstinputlisting; also try caption instead of title
}

\begin{document}

\title{Podešavanje težina neuronske mreže upotrebom optimizacionog algoritma\\ \small{Seminarski rad u okviru kursa\\Računarska inteligencija\\ Matematički fakultet}}

\author{Nikola Stamenić, Lea Petković\\ mi16177@alas.matf.bg.ac.rs, mi16163@alas.matf.bg.ac.rs}

%\date{9.~april 2015.}

\maketitle

\abstract{
U ovom tekstu je ukratko prikazana osnovna forma seminarskog rada. 

\tableofcontents

\newpage

\section{Uvod}
\label{sec:uvod}

Ovde ide neki uvod

\section{Neuronske mreže}
\label{neuronskemreze}

Neuronska mreža (eng. \emph{Artificial Neural Networks, ANN}) je sistem koji vrši mapiranje između ulaza i izlaza problema. Neuronske mreže zapravo predstavljaju parametrizovanu reprezentaciju koja se može koristiti za aproksimaciju raznih funkcija \cite{hindawi}. Matematičkom optimizacijom nekog od kriterijuma kvaliteta vrši se pronalaženje odgovarajućih parametara. 

Neuronske mreže uče informacije kroz proces treniranja u nekoliko iteracija. Kada je proces učenja završen, neuronska mreža je spremna i sposobna da klasifikuje nove informacije, predvidi ponašanje, ili aproksimira nelinearnu funkciju problema. Njena struktura sastoji se od skupa neurona, predstavljenih
funkcijama, koji su međusobno povezani sa ostalim neuronim organizovanim u slojevima. 

Struktura neuronske mreže se razlikuje po broju slojeva. Prvi sloj jeste ulazni sloj, poslednji sloj jeste izlazni, a svi slojevi između se nazivaju skrivenim
slojevima. Slojevi su međusobno potpuno povezani. Slojevi komuniciraju zahvaljujući tome što je izlaz svakog neurona, iz prethodnog sloja, povezan sa
ulazima svih neurona iz narednog sloja. Jačina veza kojom su neuroni međusobno povezani se naziva težinski faktor (eng. \emph{weight}). Najčešće ima
3 sloja.

Postoje različite vrste neuronskih mreža. Mozemo ih klasifikovati prema: broju slojeva (jednoslojne i višeslojne), vrsti veza između neurona, smeru
prostiranja informacija (neuronske mreže sa propagacijom unapred ili unazad) \cite{website}, vrsti podataka itd. 

Njihove primene su mnogobrojne, obzirom da predstavljaju najčešće primenjivanu metodu mašinskog učenja. Neke od primena su: kategorizacija teksta,
medicinska dijagnostika, prepoznavanje objekata na slikama, autonomna vožnja, igranje igara poput igara na tabli ili video igara, mašinsko prevođenje
prirodnih jezika, prepoznavanje rukom pisanih tekstova itd. 

\subsection{Keras Pajton biblioteka}
\label{subsec:keras}

Keras je biblioteka za neuronske mreže, napisana u Pajtonu. Keras radi na platformi za mašinsko učenje  \textit{TensorFlow}. Razvijena je sa ciljem da omogući brzo eksperimentisanje sa neuronskim mrežama \cite{keraswebsite}, da bude razumljiva, modularna i proširiva.

Pored, već pomenute, \textit{TensorFlow} platforme, ova biblioteka radi i na: \textit{Microsoft Cognitive Toolkit, R, Theano, ili PlaidML} platformama. Nastala je kao deo istraživanja u okviru projekta ONEIROS (eng. \emph{Open-ended Neuro-Electronic Intelligent Robot Operating System}), a njen autor i održavaoc je gugl inženjer -  François Chollet.
% \begin{verbatim}
% ščćžđ
% \end{verbatim}

\section{Optimizacija rojem čestica}
\label{subsec:pso}
U ovoj sekcije biće objašnjen sam algoritam optimizacije rojem čestica. Najviše vremena biće posvećeno originalnom algoritmu PSO, a kao i drugoj generaciji algoritma PSO (eng. \textit{The Secong Generation of PSO}), kao i novom modelu algoritma PSO 
(eng. \textit{A New Model of PSO}).


\subsection{Originalni algoritam PSO}
\label{subsec:opso}
Algoritam PSO (eng. \textit{Particle Swarm Optimization}) je metod za optimizaciju neprekidne nelinearne funkcije, koji je predložio Eberhart.
Sam algoritam je inspirisan posmatranjem socijalnog i kolektivnog ponašanja u kretanju jata ptica pri potrazi za hranom ili preživaljavanjem.
PSO je nadahnut kretanjem najboljeg člana populacije i njegovog iskustva. Metafora govori da se skup rešenja kreće prostorom pretrage sa ciljem da nađe što bolju poziciju, rešenje \cite{hindawi}.
\\
Populacijom se smatra grupa jedinki \textit{i} gde svaka predstavlja poziciju \textbf{\textbf{$x_i \in R^D$, i = 1,...,M}} u višedimenzionom prostoru.
Jedinke se evaluiraju u posebnoj funkciji optimizacije, kako bi se odredila njihova prilagođenost i sačuvala najbolja vrednost. Svaka jedinka se kreće po
prostoru pretrage u zavisnosti od funkcije brzine \textbf{$v_i$} koja u obzir uzima globalno najbolju poziciju u populaciji ($p_g \in R^D$ - socijalna
komponenta) kao i najbolju poziciju date jedinke ($p_g \in R^D$ - kognitivna komponenta). Jedinke će se kretati u svakoj iteraciji na drugu poziciju,
dok ne dostignu optimalnu poziciju. U svakom momentu \textit{t}, brzine jedinke \textit{i} se ažurira koristeći: 
\begin{center}
\textbf\textit{$v_i(t+1) = \omega v_i(t) + c_1 r_1(p_i (t) - x_i (t)) + c_2 r_2 (p_g (t) - x_i (t))$}
\end{center}
gde je $\omega$ inertna težina i obično je postavljena da varira linearno od 1 do 0 tokom iteracije, $c_1$ i $c_2$ su koeficijenti ubrzanja, $r_1$ i $r_2$
su slučajni brojevi iz uniformne (0,1) raspodele. Ubrzanje \textbf{$v_i$} je ograničeno između [$v_min, v_max$]. Ažuriranjem ubrzanja na ovaj 
način dozvoljavamo jedinki $i$ da traži najbolju oziciju \textbf{$p_i(t)$}, dok se najbolje globalno rešenje računa:\cite{hindawi}
\begin{center}
\textbf\textit{$x_i(t+1) = x_i(t) + v_i(t+1)$}
\end{center} 

\subsection{Druga generacija PSO algoritma}
\label{subsec:sgpso}
Druga generacija PSO algoritma je unapređenje originalnog PSO algoritma, koja u obzir uzima tri aspekta: lokalni optimum svih jedinki,
globalno najbolje rešenje, i novi koncept - geometrijski centar optimalne populacije. Autor knjige \textit{"{}Second Generation Particle Swarm Optimization"} objašnjava da ptice održavaju 
određenu distancu između centra jata (hrane). Jata ptica uvek ostaju u istom regionu neko vreme, tokom kojeg će centar jata ostati
nepomeren u očima jedinki. Nakon toga, jato se kreće na sledeći region, tada sve jedinke moraju održati određenu distancu sa centrom jata.

\subsection{Novi model PSO-a}
\label{subsec:nmpso}
Ovaj algoritam je predložen od strane Garoa (eng. \textit{Garro}), a on ga je bazirao na osnovu ideja drugih autora koji su predlagali unapređenje
originalnog PSO algoritma. 
Shi i Eberhart su predlagali linearno variranje inertnih težina kroz generacije, što je znatno unapredilo performanse originalnog PSO algoritma.
Yu je razvio strategiju da kada se kroz generacije globalno najbolje rešenje ne poboljšava, svaka jedinka \textit{$i$} biva izabrana sa predefinisanom
verovatnoćom a zatim dodat slučajni šum svakom vektoru brzine dimenzije $v_i$ izabrane jedinke \textit{$i$}.
Bazirano na nekim evolutivnim shemama Genetičkih Algoritama, nekoliko efektnih mutacija i ukrštanja su predložene za PSO.



\label{subsec:opso}

Vise o PSO. 


\section{Zaključak}
\label{sec:zakljucak}

Ovde pišem zaključak. 


\addcontentsline{toc}{section}{Literatura}
\appendix
\bibliography{seminarski} 
\bibliographystyle{plain}

\appendix
\section{Dodatak}
Ovde pišem dodatne stvari, ukoliko za time ima potrebe.


\end{document}
